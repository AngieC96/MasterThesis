\chapter{Conclusion}


We have succeeded in developing a method that reaches, on average, a better solution than the current one used by the technicians. However, it is possible that, using a different cost structure, these results do not remain the same. The performances of the heuristic technique and the policy gradient algorithm may increase or decrease if we change the definition of the cost. Nevertheless, it is remarkable that, even in this case, our algorithm would continue to be applicable and scalable. In fact, besides our specific results, what is important is to have developed a method that can systematically find an optimal policy in the model-based context, in which we are, since we know the consequences of our actions.

Another possible approach would be to use a model of \emph{mixed integer (linear) programming}, where only some unknown variables are required to be integers, while the others are continuous. In this case, it is possible to find an upper bound on the cost of the restoration process. What can be done is to put ourselves in the worst case, in which we have to visit \emph{every} initially disconnected substation, and compute the cost for every possible sequence of visits to the substations and for every possible position of the fault. Then, for every possible visiting path, we take the maximum cost among the different positions of the fault. As an estimate for the cost of the fault, we take the minimum cost among the ones corresponding to the different visiting sequences. In this way we can say that, for sure, we will never pay more than what we estimated, but very likely we will pay a lower price, since sometimes we will be able to reconnect more substations at once. An algorithm has been developed from third parties as a spin-off of this work, and could be expanded in future works.

In real life, though, the costs are not quite like we modeled them, they are much more random, since they depend on many factors: whether it is day or night, whether the instrumental test is applicable, the weather conditions, and, of course, many other unexpected events can happen. If we are to model more accurately these costs, first of all we would have to collect data on them, using maybe a Telegram bot connected to an \acrshort{aws} server, with which the technicians can interact while solving the fault. Then a whole other class of algorithms would be needed to solve this problem: the \emph{model-free}\index{model-free} ones, specifically designed to handle these uncertainties about the costs. This would be an interesting direction for future works, despite we will not explore it in this thesis.
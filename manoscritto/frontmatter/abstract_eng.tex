\chapter*{Abstract}
\addcontentsline{toc}{chapter}{Abstract}

The restoration of electricity after a fault is an essential task in the operation of power systems. The restoration process returns the system back to normal operation after a fault happened on the electrical line. Restoration has traditionally been performed by qualified technicians, assisted by guidelines developed by utilities using knowledge gained through experience. But now, in the era of computers and big data, it is natural to try to find a way to model this problem computationally and to solve it using more advanced techniques. Over the course of this thesis, we will both model algorithmically the current technique used by the technicians and find a new method that can exploit the information of the environment that we have available.

Our aim is to develop a new, better algorithm that can help the technicians in their decisions, and that can be used to simulate the process of restoring the operation of the power grid. We frame the problem using a partially observable Markov decision process, and we solve it using two different policy gradient methods: the policy gradient method and the natural policy gradient one. We will see that our algorithms are able, on average, to solve the fault with a lower cost with respect to what a technician would have done.
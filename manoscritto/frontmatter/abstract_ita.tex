\chapter*{Sommario} % Riassunto
\addcontentsline{toc}{chapter}{Sommario}

\setcounter{page}{1}


Il ripristino dell'energia elettrica dopo un guasto è un compito essenziale nel funzionamento dei sistemi di alimentazione. Il processo di ripristino riporta il sistema al normale funzionamento dopo che si è verificato un guasto sulla linea elettrica. Il ripristino è tradizionalmente eseguito da tecnici qualificati, assistiti da linee guida sviluppate dalle compagnie utilizzando le conoscenze acquisite attraverso l'esperienza. Ma ora, nell'era dei computer e dei \textit{big data}, è naturale cercare di trovare un modo per modellare questo problema in modo computazionale e risolverlo utilizzando tecniche più avanzate. Nel corso di questa tesi, modelleremo algoritmicamente l'attuale tecnica utilizzata dai tecnici e troveremo un nuovo metodo in grado di sfruttare le informazioni che abbiamo a disposizione sull'ambiente.

Il nostro obiettivo è sviluppare un nuovo e migliore algoritmo che possa aiutare i tecnici nelle loro decisioni e che possa essere utilizzato per simulare il processo di ripristino del funzionamento della rete elettrica. Abbiamo inquadrato il problema utilizzando un processo decisionale di Markov parzialmente osservabile e lo abbiamo risolto utilizzando due diversi metodi di \textit{policy gradient}: il \textit{policy gradient} e il \textit{natural policy gradient}. Vedremo che i nostri algoritmi sono in grado, in media, di risolvere il guasto con un costo inferiore rispetto a quanto avrebbe fatto un tecnico.